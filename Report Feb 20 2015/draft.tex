%% bare_conf.tex
%% V1.4
%% 2012/12/27
%% by Michael Shell
%% See:
%% http://www.michaelshell.org/
%% for current contact information.
%%
%% This is a skeleton file demonstrating the use of IEEEtran.cls
%% (requires IEEEtran.cls version 1.8 or later) with an IEEE conference paper.
%%
%% Support sites:
%% http://www.michaelshell.org/tex/ieeetran/
%% http://www.ctan.org/tex-archive/macros/latex/contrib/IEEEtran/
%% and
%% http://www.ieee.org/

%%*************************************************************************
%% Legal Notice:
%% This code is offered as-is without any warranty either expressed or
%% implied; without even the implied warranty of MERCHANTABILITY or
%% FITNESS FOR A PARTICULAR PURPOSE! 
%% User assumes all risk.
%% In no event shall IEEE or any contributor to this code be liable for
%% any damages or losses, including, but not limited to, incidental,
%% consequential, or any other damages, resulting from the use or misuse
%% of any information contained here.
%%
%% All comments are the opinions of their respective authors and are not
%% necessarily endorsed by the IEEE.
%%
%% This work is distributed under the LaTeX Project Public License (LPPL)
%% ( http://www.latex-project.org/ ) version 1.3, and may be freely used,
%% distributed and modified. A copy of the LPPL, version 1.3, is included
%% in the base LaTeX documentation of all distributions of LaTeX released
%% 2003/12/01 or later.
%% Retain all contribution notices and credits.
%% ** Modified files should be clearly indicated as such, including  **
%% ** renaming them and changing author support contact information. **
%%
%% File list of work: IEEEtran.cls, IEEEtran_HOWTO.pdf, bare_adv.tex,
%%                    bare_conf.tex, bare_jrnl.tex, bare_jrnl_compsoc.tex,
%%                    bare_jrnl_transmag.tex
%%*************************************************************************

% *** Authors should verify (and, if needed, correct) their LaTeX system  ***
% *** with the testflow diagnostic prior to trusting their LaTeX platform ***
% *** with production work. IEEE's font choices can trigger bugs that do  ***
% *** not appear when using other class files.                            ***
% The testflow support page is at:
% http://www.michaelshell.org/tex/testflow/



% Note that the a4paper option is mainly intended so that authors in
% countries using A4 can easily print to A4 and see how their papers will
% look in print - the typesetting of the document will not typically be
% affected with changes in paper size (but the bottom and side margins will).
% Use the testflow package mentioned above to verify correct handling of
% both paper sizes by the user's LaTeX system.
%
% Also note that the "draftcls" or "draftclsnofoot", not "draft", option
% should be used if it is desired that the figures are to be displayed in
% draft mode.
%
\documentclass[12pt,letter,final]{article}
% Add the compsoc option for Computer Society conferences.
%
% If IEEEtran.cls has not been installed into the LaTeX system files,
% manually specify the path to it like:
\usepackage[latin1]{inputenc}
\usepackage{amsfonts}
\usepackage{amssymb}
\usepackage{amsthm}
\usepackage{fullpage}
\usepackage{setspace}
\usepackage{graphicx}
%\usepackage[pdftex]{graphicx}
\usepackage{psfrag}
\usepackage{color}
\usepackage{epsfig}
\usepackage{appendix}
\usepackage{caption}
\usepackage{cite}
\usepackage{ifpdf}
\usepackage[cmex10]{amsmath}
\usepackage{algorithmic}
\usepackage{array}
\usepackage{stfloats}
\usepackage{url}
\usepackage{fixltx2e}
\usepackage{setspace} 
\usepackage{diagbox}





% Some very useful LaTeX packages include:
% (uncomment the ones you want to load)


% *** MISC UTILITY PACKAGES ***
%
%\usepackage{ifpdf}
% Heiko Oberdiek's ifpdf.sty is very useful if you need conditional
% compilation based on whether the output is pdf or dvi.
% usage:
% \ifpdf
%   % pdf code
% \else
%   % dvi code
% \fi
% The latest version of ifpdf.sty can be obtained from:
% http://www.ctan.org/tex-archive/macros/latex/contrib/oberdiek/
% Also, note that IEEEtran.cls V1.7 and later provides a builtin
% \ifCLASSINFOpdf conditional that works the same way.
% When switching from latex to pdflatex and vice-versa, the compiler may
% have to be run twice to clear warning/error messages.






% *** CITATION PACKAGES ***
%
%\usepackage{cite}
% cite.sty was written by Donald Arseneau
% V1.6 and later of IEEEtran pre-defines the format of the cite.sty package
% \cite{} output to follow that of IEEE. Loading the cite package will
% result in citation numbers being automatically sorted and properly
% "compressed/ranged". e.g., [1], [9], [2], [7], [5], [6] without using
% cite.sty will become [1], [2], [5]--[7], [9] using cite.sty. cite.sty's
% \cite will automatically add leading space, if needed. Use cite.sty's
% noadjust option (cite.sty V3.8 and later) if you want to turn this off
% such as if a citation ever needs to be enclosed in parenthesis.
% cite.sty is already installed on most LaTeX systems. Be sure and use
% version 4.0 (2003-05-27) and later if using hyperref.sty. cite.sty does
% not currently provide for hyperlinked citations.
% The latest version can be obtained at:
% http://www.ctan.org/tex-archive/macros/latex/contrib/cite/
% The documentation is contained in the cite.sty file itself.






% *** GRAPHICS RELATED PACKAGES ***
%
%\ifCLASSINFOpdf
  % \usepackage[pdftex]{graphicx}
  % declare the path(s) where your graphic files are
  % \graphicspath{{../pdf/}{../jpeg/}}
  % and their extensions so you won't have to specify these with
  % every instance of \includegraphics
  % \DeclareGraphicsExtensions{.pdf,.jpeg,.png}
%\else
  % or other class option (dvipsone, dvipdf, if not using dvips). graphicx
  % will default to the driver specified in the system graphics.cfg if no
  % driver is specified.
  % \usepackage[dvips]{graphicx}
  % declare the path(s) where your graphic files are
  % \graphicspath{{../eps/}}
  % and their extensions so you won't have to specify these with
  % every instance of \includegraphics
  % \DeclareGraphicsExtensions{.eps}
%\fi
% graphicx was written by David Carlisle and Sebastian Rahtz. It is
% required if you want graphics, photos, etc. graphicx.sty is already
% installed on most LaTeX systems. The latest version and documentation
% can be obtained at: 
% http://www.ctan.org/tex-archive/macros/latex/required/graphics/
% Another good source of documentation is "Using Imported Graphics in
% LaTeX2e" by Keith Reckdahl which can be found at:
% http://www.ctan.org/tex-archive/info/epslatex/
%
% latex, and pdflatex in dvi mode, support graphics in encapsulated
% postscript (.eps) format. pdflatex in pdf mode supports graphics
% in .pdf, .jpeg, .png and .mps (metapost) formats. Users should ensure
% that all non-photo figures use a vector format (.eps, .pdf, .mps) and
% not a bitmapped formats (.jpeg, .png). IEEE frowns on bitmapped formats
% which can result in "jaggedy"/blurry rendering of lines and letters as
% well as large increases in file sizes.
%
% You can find documentation about the pdfTeX application at:
% http://www.tug.org/applications/pdftex





% *** MATH PACKAGES ***
%
%\usepackage[cmex10]{amsmath}
% A popular package from the American Mathematical Society that provides
% many useful and powerful commands for dealing with mathematics. If using
% it, be sure to load this package with the cmex10 option to ensure that
% only type 1 fonts will utilized at all point sizes. Without this option,
% it is possible that some math symbols, particularly those within
% footnotes, will be rendered in bitmap form which will result in a
% document that can not be IEEE Xplore compliant!
%
% Also, note that the amsmath package sets \interdisplaylinepenalty to 10000
% thus preventing page breaks from occurring within multiline equations. Use:
%\interdisplaylinepenalty=2500
% after loading amsmath to restore such page breaks as IEEEtran.cls normally
% does. amsmath.sty is already installed on most LaTeX systems. The latest
% version and documentation can be obtained at:
% http://www.ctan.org/tex-archive/macros/latex/required/amslatex/math/





% *** SPECIALIZED LIST PACKAGES ***
%
%\usepackage{algorithmic}
% algorithmic.sty was written by Peter Williams and Rogerio Brito.
% This package provides an algorithmic environment fo describing algorithms.
% You can use the algorithmic environment in-text or within a figure
% environment to provide for a floating algorithm. Do NOT use the algorithm
% floating environment provided by algorithm.sty (by the same authors) or
% algorithm2e.sty (by Christophe Fiorio) as IEEE does not use dedicated
% algorithm float types and packages that provide these will not provide
% correct IEEE style captions. The latest version and documentation of
% algorithmic.sty can be obtained at:
% http://www.ctan.org/tex-archive/macros/latex/contrib/algorithms/
% There is also a support site at:
% http://algorithms.berlios.de/index.html
% Also of interest may be the (relatively newer and more customizable)
% algorithmicx.sty package by Szasz Janos:
% http://www.ctan.org/tex-archive/macros/latex/contrib/algorithmicx/




% *** ALIGNMENT PACKAGES ***
%
%\usepackage{array}
% Frank Mittelbach's and David Carlisle's array.sty patches and improves
% the standard LaTeX2e array and tabular environments to provide better
% appearance and additional user controls. As the default LaTeX2e table
% generation code is lacking to the point of almost being broken with
% respect to the quality of the end results, all users are strongly
% advised to use an enhanced (at the very least that provided by array.sty)
% set of table tools. array.sty is already installed on most systems. The
% latest version and documentation can be obtained at:
% http://www.ctan.org/tex-archive/macros/latex/required/tools/


% IEEEtran contains the IEEEeqnarray family of commands that can be used to
% generate multiline equations as well as matrices, tables, etc., of high
% quality.




% *** SUBFIGURE PACKAGES ***
%\ifCLASSOPTIONcompsoc
%  \usepackage[caption=false,font=normalsize,labelfont=sf,textfont=sf]{subfig}
%\else
%  \usepackage[caption=false,font=footnotesize]{subfig}
%\fi
% subfig.sty, written by Steven Douglas Cochran, is the modern replacement
% for subfigure.sty, the latter of which is no longer maintained and is
% incompatible with some LaTeX packages including fixltx2e. However,
% subfig.sty requires and automatically loads Axel Sommerfeldt's caption.sty
% which will override IEEEtran.cls' handling of captions and this will result
% in non-IEEE style figure/table captions. To prevent this problem, be sure
% and invoke subfig.sty's "caption=false" package option (available since
% subfig.sty version 1.3, 2005/06/28) as this is will preserve IEEEtran.cls
% handling of captions.
% Note that the Computer Society format requires a larger sans serif font
% than the serif footnote size font used in traditional IEEE formatting
% and thus the need to invoke different subfig.sty package options depending
% on whether compsoc mode has been enabled.
%
% The latest version and documentation of subfig.sty can be obtained at:
% http://www.ctan.org/tex-archive/macros/latex/contrib/subfig/




% *** FLOAT PACKAGES ***
%
%\usepackage{fixltx2e}
% fixltx2e, the successor to the earlier fix2col.sty, was written by
% Frank Mittelbach and David Carlisle. This package corrects a few problems
% in the LaTeX2e kernel, the most notable of which is that in current
% LaTeX2e releases, the ordering of single and double column floats is not
% guaranteed to be preserved. Thus, an unpatched LaTeX2e can allow a
% single column figure to be placed prior to an earlier double column
% figure. The latest version and documentation can be found at:
% http://www.ctan.org/tex-archive/macros/latex/base/


%\usepackage{stfloats}
% stfloats.sty was written by Sigitas Tolusis. This package gives LaTeX2e
% the ability to do double column floats at the bottom of the page as well
% as the top. (e.g., "\begin{figure*}[!b]" is not normally possible in
% LaTeX2e). It also provides a command:
%\fnbelowfloat
% to enable the placement of footnotes below bottom floats (the standard
% LaTeX2e kernel puts them above bottom floats). This is an invasive package
% which rewrites many portions of the LaTeX2e float routines. It may not work
% with other packages that modify the LaTeX2e float routines. The latest
% version and documentation can be obtained at:
% http://www.ctan.org/tex-archive/macros/latex/contrib/sttools/
% Do not use the stfloats baselinefloat ability as IEEE does not allow
% \baselineskip to stretch. Authors submitting work to the IEEE should note
% that IEEE rarely uses double column equations and that authors should try
% to avoid such use. Do not be tempted to use the cuted.sty or midfloat.sty
% packages (also by Sigitas Tolusis) as IEEE does not format its papers in
% such ways.
% Do not attempt to use stfloats with fixltx2e as they are incompatible.
% Instead, use Morten Hogholm'a dblfloatfix which combines the features
% of both fixltx2e and stfloats:
%
% \usepackage{dblfloatfix}
% The latest version can be found at:
% http://www.ctan.org/tex-archive/macros/latex/contrib/dblfloatfix/




% *** PDF, URL AND HYPERLINK PACKAGES ***
%
%\usepackage{url}
% url.sty was written by Donald Arseneau. It provides better support for
% handling and breaking URLs. url.sty is already installed on most LaTeX
% systems. The latest version and documentation can be obtained at:
% http://www.ctan.org/tex-archive/macros/latex/contrib/url/
% Basically, \url{my_url_here}.




% *** Do not adjust lengths that control margins, column widths, etc. ***
% *** Do not use packages that alter fonts (such as pslatex).         ***
% There should be no need to do such things with IEEEtran.cls V1.6 and later.
% (Unless specifically asked to do so by the journal or conference you plan
% to submit to, of course. )


% correct bad hyphenation here
%\hyphenation{op-tical net-works semi-conduc-tor}



%
% paper title
% can use linebreaks \\ within to get better formatting as desired
% Do not put math or special symbols in the title.
\title{Report}


% author names and affiliations
% use a multiple column layout for up to three different
% affiliations
%\author{\IEEEauthorblockN{Tianpei Chen}
%\IEEEauthorblockA{ID: 25504946\\Department of Electrical and\\Computer Engineering\\
%McGill University\\
%Montreal, Quebec}}
\author{Tianpei Chen\\
 Department of Electrical and Computer Engineering\\}

% conference papers do not typically use \thanks and this command
% is locked out in conference mode. If really needed, such as for
% the acknowledgment of grants, issue a \IEEEoverridecommandlockouts
% after \documentclass

% for over three affiliations, or if they all won't fit within the width
% of the page, use this alternative format:
% 
%\author{\IEEEauthorblockN{Michael Shell\IEEEauthorrefmark{1},
%Homer Simpson\IEEEauthorrefmark{2},
%James Kirk\IEEEauthorrefmark{3}, 
%Montgomery Scott\IEEEauthorrefmark{3} and
%Eldon Tyrell\IEEEauthorrefmark{4}}
%\IEEEauthorblockA{\IEEEauthorrefmark{1}School of Electrical and Computer Engineering\\
%Georgia Institute of Technology,
%Atlanta, Georgia 30332--0250\\ Email: see http://www.michaelshell.org/contact.html}
%\IEEEauthorblockA{\IEEEauthorrefmark{2}Twentieth Century Fox, Springfield, USA\\
%Email: homer@thesimpsons.com}
%\IEEEauthorblockA{\IEEEauthorrefmark{3}Starfleet Academy, San Francisco, California 96678-2391\\
%Telephone: (800) 555--1212, Fax: (888) 555--1212}
%\IEEEauthorblockA{\IEEEauthorrefmark{4}Tyrell Inc., 123 Replicant Street, Los Angeles, California 90210--4321}}




% use for special paper notices
%\IEEEspecialpapernotice{(Invited Paper)}



\begin{document}
% make the title area
\begin{spacing}{2.5}
\maketitle

% As a general rule, do not put math, special symbols or citations
% in the abstract
%\begin{abstract}

%\end{abstract}

% no keywords




% For peer review papers, you can put extra information on the cover
% page as needed:
% \ifCLASSOPTIONpeerreview
% \begin{center} \bfseries EDICS Category: 3-BBND \end{center}
% \fi
%
% For peerreview papers, this IEEEtran command inserts a page break and
% creates the second title. It will be ignored for other modes.
%\IEEEpeerreviewmaketitle



\section{System Model}\label{system}
% no \IEEEPARstart
%This demo file is intended to serve as a ``starter file''
%for IEEE conference papers produced under \LaTeX\ using
%IEEEtran.cls version 1.8 and later.
% You must have at least 2 lines in the paragraph with the drop letter
% (should never be an issue)
%I wish you the best of success.
%\hfill mds
%\hfill December 27, 2012
We consider a complex uncoded spatial multiplexing MIMO system with $N_r$ receive and $N_t$ transmit antennas, $N_{r}\geq N_{t}$, over a flat fading channel. Using a discrete time model, $\mathbf{y}\in\mathbb{C}^{N_{r}\times 1}$ is the received symbol vector written as:
\begin{equation}
\mathbf{y}=\mathbf{H}\mathbf{s}+\mathbf{n},   \label{formula 1}
\end{equation}
where $\mathbf{s}\in \mathbb{C}^{N_{t}\times 1}$ is the transmitted symbol vector, with components that are mutually independent and taken from a signal constellation $\mathbb{O}$ (4-QAM, 16-QAM, 64-QAM) of size $M$. The possible transmitted symbol vectors $\mathbf{s}\in \mathbb{O}^{N_{t}}$, satisfy $\mathbb{E}[\mathbf{s}\mathbf{s}^{H}]=\mathbf{I}_{N_t}E_{s}$, where $E_{s}$ denotes the symbol average energy, and $\mathbb{E}[\cdot]$ denotes the expectation operation. Furthermore $\mathbf{H}\in \mathbb{C}^{N_{r}\times N_{t}}$ denotes the Rayleigh fading channel propagation matrix with independent identically distributed (i.i.d) circularly symmetric complex Gaussian components of zero mean and unit variance. Finally, $\mathbf{n}\in \mathbb{C}^{N_{r}\times 1}$ is the additive white Gaussian noise (AWGN) vector with zero mean components and $\mathbb{E}[\mathbf{n}\mathbf{n}^{H}]=\mathbf{I}_{N_{r}}N_{0}$, where $N_{0}$ denotes the noise power spectrum density, and hence $\frac{E_{s}}{N_{0}}$ is the signal to noise ratio (SNR). 

Assume the receiver has perfect channel state information (CSI), meaning that $ \mathbf{H}$ is known, as well as the SNR. The task of the MIMO decoder is to recover $\mathbf{s}$ based on $\mathbf{y}$ and $\mathbf{H}$.


% An example of a floating figure using the graphicx package.
% Note that \label must occur AFTER (or within) \caption.

% For figures, \caption should occur after the \includegraphics.
% Note that IEEEtran v1.7 and later has special internal code that
% is designed to preserve the operation of \label within \caption
% even when the captionsoff option is in effect. However, because
% of issues like this, it may be the safest practice to put all your
% \label just after \caption rather than within \caption{}.
%
% Reminder: the "draftcls" or "draftclsnofoot", not "draft", class
% option should be used if it is desired that the figures are to be
% displayed while in draft mode.
%
%\begin{figure}[!t]
%\centering
%\includegraphics[width=2.5in]{myfigure}
% where an .eps filename suffix will be assumed under latex, 
% and a .pdf suffix will be assumed for pdflatex; or what has been declared
% via \DeclareGraphicsExtensions.
%\caption{Simulation Results.}
%\label{fig_sim}
%\end{figure}

% Note that IEEE typically puts floats only at the top, even when this
% results in a large percentage of a column being occupied by floats.


% An example of a double column floating figure using two subfigures.
% (The subfig.sty package must be loaded for this to work.)
% The subfigure \label commands are set within each subfloat command,
% and the \label for the overall figure must come after \caption.
% \hfil is used as a separator to get equal spacing.
% Watch out that the combined width of all the subfigures on a 
% line do not exceed the text width or a line break will occur.
%
%\begin{figure*}[!t]
%\centering
%\subfloat[Case I]{\includegraphics[width=2.5in]{box}%
%\label{fig_first_case}}
%\hfil
%\subfloat[Case II]{\includegraphics[width=2.5in]{box}%
%\label{fig_second_case}}
%\caption{Simulation results.}
%\label{fig_sim}
%\end{figure*}
%
% Note that often IEEE papers with subfigures do not employ subfigure
% captions (using the optional argument to \subfloat[]), but instead will
% reference/describe all of them (a), (b), etc., within the main caption.


% An example of a floating table. Note that, for IEEE style tables, the 
% \caption command should come BEFORE the table. Table text will default to
% \footnotesize as IEEE normally uses this smaller font for tables.
% The \label must come after \caption as always.
%
%\begin{table}[!t]
%% increase table row spacing, adjust to taste
%\renewcommand{\arraystretch}{1.3}
% if using array.sty, it might be a good idea to tweak the value of
% \extrarowheight as needed to properly center the text within the cells
%\caption{An Example of a Table}
%\label{table_example}
%\centering
%% Some packages, such as MDW tools, offer better commands for making tables
%% than the plain LaTeX2e tabular which is used here.
%\begin{tabular}{|c||c|}
%\hline
%One & Two\\
%\hline
%Three & Four\\
%\hline
%\end{tabular}
%\end{table}
\section{Modification of Orthogonality Deficiency}
Original definition of orthogonality deficiency:
\begin{equation}
\phi_{od}=1-\frac{det(\mathbf{W})}{\prod_{i=1}^{N_{t}}||\mathbf{h}_{i}||^{2}},
\label{formula1}
\end{equation}
where $\mathbf{W}=\mathbf{H}^{H}\mathbf{H}$ denotes Wishart matrix, $\mathbf{h}_{i}$ denotes the $i$ th column of $\mathbf{H}$, $det(\cdot)$ denotes determinant operation, $||\cdot||^{2}$ denotes 2-norm operation.
In (\ref{formula1}), $||\mathbf{h}_{i}||^{2}=\sum_{i=1}^{N_{t}}|\mathbf{H}_{ij}|^{2}$, $\mathbf{H}_{ij}$ denotes the component of $\mathbf{H}$ at $i$ th row and $j$ th column. $\mathbf{H}_{ij}\sim Rayleigh(1/\sqrt{2})$, therefore $||\mathbf{h}_{i}||^{2}\sim\Gamma(N_{r},1)$\cite{papoulis1996stochastic}. $\Gamma(k,\theta)$ denotes Gamma distribution, with $k$ degrees of freedom. Furthermore, we have:
\begin{equation}
2||\mathbf{h}_{i}||^{2}\sim Gamma(N_{r},2)\sim\chi^{2}_{2N_{r}},
\label{formula2}
\end{equation} 
$\chi^{2}_{k}$ denotes chi-square distribution with $k$ degrees of freedom. Because $\ln(\chi^{2})$ converges much faster than $\chi^{2}$\cite{shoemaker2003fixing}\cite{bartlett1946statistical} as well as for simplicity, (\ref{formula1}) can be changed to:
\begin{equation}
\phi_{om}=\frac{2^{N_{t}}det(\mathbf{W})}{\prod_{i=1}^{N_{t}}2||\mathbf{h}_{i}||^{2}}
\longrightarrow \frac{N_{t}\ln{2}+\ln{det(\mathbf{W})}}{\sum_{i=1}^{N_{t}}\ln{2}||\mathbf{h}_{i}||^{2}},
\label{formula3}
\end{equation}
$\phi_{om}$ in (\ref{formula3}) is defined as Orthogonality Measure. Based on Hadamard's inequality ($\prod_{i=1}^{N_{t}}||\mathbf{h}_{i}||\geq det(\mathbf{H})$) $\phi_{om}\in [0,1]$. If $\phi_{om}$ is more closer to 1, $\mathbf{H}$ is closer to orthogonal matrix. 
\section{Derivation of Marginal Probability Density Functions (PDFs)}
First we consider the marginal PDFs of the components in (\ref{formula3}), define 
\begin{eqnarray}
V=\sum_{i=1}^{N_{t}}\ln{2||\mathbf{h}_{i}||^{2}},\\
U=N_{t}\ln{2}+\ln{det{\mathbf{W}}},
\label{formula4}
\end{eqnarray}
where $V$ is the sum of components $\ln{2||\mathbf{h}_{i}||^{2}}\quad i\in [1,N_{t}]$. Since each component converges to normality rapidly, it can be easily proved that $V$ converges to normality.

Considering $U$, $\mathbf{W}=\mathbf{H}^{H}\mathbf{H}$, do $QR$ factorization:
\begin{equation}
\mathbf{H}=\mathbf{Q}\mathbf{R},
\label{formula5} 
 \end{equation}
where $\mathbf{Q}\in\mathbb{C}^{N_{r}\times N_{t}}$ is a unitary matrix and $\mathbf{R}\in\mathbb{C}^{N_{t}\times N_{t}}$ is the upper triangular matrix. Using (\ref{formula5}), we have $\mathbf{W}=\mathbf{R}^{H}\mathbf{R}$. $r_{ii}$ denotes the $i$ th diagonal component of $\mathbf{R}$, thus $U$ can be rewritten as:
\begin{equation}
U=N_{t}\ln{2}+\ln(det{\mathbf{R}^{H}\mathbf{R}})=N_{t}\ln{2}+\ln{det(\mathbf{R}^{H})det(\mathbf{R})}\\=N_{t}\ln{2}+\ln{\prod_{i=1}^{N_{t}}r_{ii}^{H}\prod_{i=1}^{N_{t}}r_{ii}}=N_{t}\ln{2}+\sum_{i=1}^{N_{t}}\ln{|r_{ii}|^{2}}.
\label{formula6}
\end{equation}    
The next step that can be take into account is to find the distribution of $|r_{ii}|^{2}$.

An alternative is to consider Wishart distribution of $det(\mathbf{W})$.

\section{Derivation of Probability of Orthogonality Measure}
An alternative modification of (\ref{formula3}) can be written as:
\begin{equation}
\phi_{om}=\frac{\prod_{i=1}^{N_{t}}|r_{ii}|^{2}}{\prod_{i=1}^{N_{t}}||\mathbf{h}_{i}||^{2}},
\label{formula7}
\end{equation}
Take the logarithm of (\ref{formula7}), we have
\begin{equation}
\log{\phi_{om}}=\sum_{i=1}^{N_{t}}\log{\frac{|r_{ii}|^{2}}{||\mathbf{h}_{i}||^{2}}}.
\label{formula8}
\end{equation}  
Notice that $\mathbf{R}$ can be viewed as the Cholesky factorization of $\mathbf{W}$. Based on Cholesky factorization, we have $||\mathbf{h}_{i}||^{2}=\sum^{i-1}_{j=1}|r_{ji}|^{2}+|r_{ii}|^{2}$. Thus (\ref{formula8}) can be rewritten as:
\begin{equation}
\log{\phi_{om}}=\sum_{i=1}^{N_{t}}\log{\frac{1}{\sum^{i-1}_{j=1}|r_{ji}|^{2}/|r_{ii}|^{2}+1}}.
\label{formula9}
\end{equation}  
From (\ref{formula9}), the lattice reduction we proposed should aim to reduce the component $\sum^{i-1}_{j=1}|r_{ji}|^{2}/|r_{ii}|^2$.
% conference papers do not normally have an appendix


% use section* for acknowledgement
%\section*{Acknowledgment}
\section{Derivation of Logarithmic Expectation of Orthogonality Measurement}
Taking the logarithmic of $\phi_{om}$ in (\ref{formula3}), we have 
\begin{equation}
\ln(\phi_{om})=N_{t}\ln(2)+\ln(det(\mathbf{W}))-\sum_{i=1}^{N_{t}}\ln(2||\mathbf{h}_{i}||^{2}),
\label{formula10}
\end{equation}
taking expectation, we have 
\begin{equation}
\mathbb{E}[\ln(\phi_{om})]=N_{t}\ln(2)+\mathbb{E}[\ln(det(\mathbf{W}))]-\sum_{i=1}^{N_{t}}\mathbb{E}[\ln(2||\mathbf{h}_{i}||^{2})].
\label{formula11}
\end{equation}
Consider Rayleigh fading channel, $\mathbf{H}=[\mathbf{h}_{1},\mathbf{h}_{2},\cdots \mathbf{h}_{N_{t}}]$, where $\mathbf{h}_{i}$ denotes the $i$ th column of $\mathbf{H}$, because each component of $\mathbf{H}$ is zero mean, and $\mathbf{h}_{i}$ are mutually independent. Therefore, $\mathbf{W}=\mathbf{H}^{H}\mathbf{H}\sim \mathbb{C}W(n, \mathbf{\Sigma})$ denotes complex Wishart distribution with $n$ degrees of freedom and covariance matrix $\mathbf{\Sigma}$. The logarithmic expectation of $\mathbf{W}$ can be rewritten as
\begin{equation}
\mathbb{E}[\ln(det(\mathbf{W}))]=\frac{\tilde{\Gamma}^{'}_{N_{t}}(N_{r})}{\tilde{\Gamma}_{N_{t}}(N_{r})}=\sum_{i=1}^{N_{t}}\psi(N_{r}-i+1),
\label{formula12}
\end{equation}
where $\tilde{\Gamma}_{m}(n)$ denotes the multivariate Gamma function. Proof: see Appendix A.

Because the logarithmic expectation of a Gamma distribution variable can be written as:
\begin{equation}
\mathbb{E}[\ln(Gamma(n,\theta))]=\psi(n)+\ln(\theta),
\end{equation}
where $\psi(n)$ denotes Digamma function. Thus according to (\ref{formula2}), we have:
\begin{equation}
\mathbb{E}[\ln(2||\mathbf{h}_{i}||^{2})]=\psi(N_{r})+\ln(2).
\label{formula14}
\end{equation}
Proof: see Appendix B
Based on (\ref{formula11})(\ref{formula12})(\ref{formula14}), The logarithmic expectation of $\phi_{om}$ can be written as:
\begin{eqnarray}
\nonumber
\mathbb{E}[\ln(\phi_{om})]&=&N_{t}\ln(2)+\sum_{i=1}^{N_{t}}\psi(N_{r}-i+1)-N_{t}\psi(N_{r})-N_{t}\ln(2)\\
&=& \sum_{i=1}^{N_{t}}\psi(N_{r}-i+1)-N_{t}\psi(N_{r})
\label{formula15}
\end{eqnarray}
\section{Derivation of Joint PDF of Logarithmic Orthogonality Measurement}
Recall (\ref{formula7}), because $||h_{i}||^{2}=\sum_{j\leq i}|r_{ij}|^{2}$, we can rewrite (\ref{formula7}) as
\begin{equation}
\phi_{om}=\prod_{i=1}^{N_{t}}\frac{|r_{ii}|^{2}}{|r_{ii}|^{2}+\sum_{j<i}|r_{ji}|^{2}}.
\end{equation}
All the components in $\mathbf{R}$ are independently distributed and $r_{ji}\sim \mathbb{C}N(0,1)$, $|r_{ii}|^{2}\sim Gamma((N_{r}-i+1),1)$ \cite{nagar2011expectations}. Because $|r_{ij}|$ are standard complex Gaussian, $\sum_{j<i}|r_{ij}|^{2}\sim Gamma(i-1, 1)$. Defining $\alpha_{i}=\sum_{j<i}|r_{ij}|^{2}$ and $\beta_{i}=|r_{ii}|^{2}$, $\alpha_{i}$ and $\beta_{i}$ are mutually independent, therefore (\ref{formula9}) can be rewritten to 
\begin{equation}
\phi_{om}=\prod_{i=1}^{N_{t}}\frac{\beta_{i}}{\alpha_{i}+\beta_{i}},
\label{formula21}
\end{equation}
From \cite{gupta2004handbook}, if $X\sim Gamma(k_{1},\theta)$ and $Y\sim Gamma(k_{2},\theta)$, then $\frac{X}{X+Y}\sim B(k_{1},k_{2})$, where $B$ denotes Beta distribution. Therefore $\frac{\beta_{i}}{\beta_{i}+\alpha_{i}}\sim B(k^{i}_{1}, k^{i}_{2})$, where $k^{i}_{1}=N_{r}-i+1$, $k^{i}_{2}=i-1$. we define $\eta_{i}=\frac{\beta_{i}}{\beta_{i}+\alpha_{i}}$, it is obvious that $\eta_{i}$ are independently distributed. Based on (\ref{formula21}), we have 
\begin{equation}
\phi_{om}=\prod_{i=1}^{N_{t}}\eta_{i}.
\label{formula22}
\end{equation}
Therefore the density function of $\phi_{om}$ can be defined as
\begin{equation}
f_{\phi_{om}}(x)=\prod_{i=0}^{N_{t}}f_{\eta_{i}}(l_{i}),
\label{formula23}
\end{equation}
where $f_{\eta_{i}}(l_{i})=\frac{1}{\mathbb{B}(k^{i}_{1},k^{i}_{2})}l_{i}^{k^{i}_{1}-1}(1-l_{i})^{k^{i}_{2}-1}$, $\mathbb{B}(\alpha, \beta)$ denotes Beta function.
Consider logarithmic expectation of $\phi_{om}$, we have
\begin{equation}
E[\ln(\phi_{om})]=\sum_{i=1}^{N_{t}}E[\ln(\eta_{i})],
\label{formula24}
\end{equation}
where $E[\ln(\eta_{i})]=\psi(k^{i}_{1})-\psi(k^{i}_{1}+k^{i}_{2})$, thus we have 
\begin{equation}
E[\ln(\phi_{om})]=\sum_{i=1}^{N_{t}}\psi(N_{r}-i+1)-N_{t}\psi(N_{r}).
\label{formula25}
\end{equation}
we can find (\ref{formula25}) is consistent with (\ref{formula15}). 
% trigger a \newpage just before the given reference
% number - used to balance the columns on the last page
% adjust value as needed - may need to be readjusted if
% the document is modified later
%\IEEEtriggeratref{8}
% The "triggered" command can be changed if desired:
%\IEEEtriggercmd{\enlargethispage{-5in}}

% references section

% can use a bibliography generated by BibTeX as a .bbl file
% BibTeX documentation can be easily obtained at:
% http://www.ctan.org/tex-archive/biblio/bibtex/contrib/doc/
% The IEEEtran BibTeX style support page is at:
% http://www.michaelshell.org/tex/ieeetran/bibtex/
%\bibliographystyle{IEEEtran}
% argument is your BibTeX string definitions and bibliography database(s)
%\bibliography{IEEEabrv,../bib/paper}
%
% <OR> manually copy in the resultant .bbl file
% set second argument of \begin to the number of references
% (used to reserve space for the reference number labels box)
\begin{appendices}
\section{Appendix A}
Let $\mathbf{A}\in \mathbb{C}^{m\times m}$, $A\sim \mathbb{C}W(n, \mathbf{\Sigma})$, $\mathbb{C}W(n, \mathbf{\Sigma})$ denotes complex Wishart distribution with $n$ degrees of freedom and covariance matrix $\mathbf{\Sigma}$. It is obvious $\mathbf{A}$ is Hermition positive definite matrix, $\mathbf{A}=\mathbf{A}^{H}>0$.

The p.d.f function of $\mathbf{A}$ can be written as\cite{nagar2011expectations}:
\begin{equation}
f(A)=\{\tilde{\Gamma}_{m}(n)det(\mathbf{\Sigma})^{n} \}^{-1}det(A)^{n-m}etr(-\mathbf{\Sigma}^{-1}\mathbf{A}),
\label{Appendequa1}
\end{equation}
where $\tilde{\Gamma}_{m}(\beta)$ denotes multivariate complex Gamma function defined by:
\begin{equation}
\tilde{\Gamma}_{m}(\beta)=\pi^{\frac{m(m-1)}{2}}\prod_{i=1}^{m}\Gamma(\beta-i+1)\quad Re(\beta)>m-1.
\label{Appendequa2}
\end{equation}
Furthermore,from \cite{nagar2011expectations}, we have 

\begin{equation}
\tilde{\Gamma}_{m}(\beta)=\int_{\mathbf{X}=\mathbf{X}^{H}>0}etr(-\mathbf{X})det(\mathbf{X})^{\beta-m}d
\mathbf{X} \quad Re(\beta)>m-1.
\label{Appendequa3}
\end{equation}
We derive logarithmic expectation of $det(\mathbf{A})$
\begin{eqnarray}
\nonumber
E[\ln(det(\mathbf{A}))]&=&\int_{\mathbf{A}=\mathbf{A}^{H}>0}\ln(det(\mathbf{A}))f(\mathbf{A})d\mathbf{A}\\
\nonumber
&=&\int_{\mathbf{A}=\mathbf{A}^{H}>0}\ln(det(\mathbf{A}))\{\tilde{\Gamma}_{m}(n)det(\mathbf{\Sigma})^{n} \}^{-1}det(\mathbf{A})^{n-m}etr(-\mathbf{\Sigma}^{-1}\mathbf{A})d\mathbf{A}\\
\nonumber
&=&\frac{det(\mathbf{\Sigma})^{-n}}{\tilde{\Gamma}_{m}(n)}\int_{\mathbf{A}=\mathbf{A}^{H}>0}\ln(det(\mathbf{A}))det(\mathbf{A})^{n-m}etr(-\mathbf{\Sigma}^{-1}\mathbf{A})d\mathbf{A},
\label{Appendequa4}
\end{eqnarray}
if $\mathbf{\sum}=\mathbf{I}$, \ref{Appendequa4} can be written as 
\begin{equation}
E[\ln(det(\mathbf{A}))]=\frac{1}{\tilde{\Gamma}_{m}(n)}\int_{\mathbf{A}=\mathbf{A}^{H}>0}\ln(det(\mathbf{A}))det(\mathbf{A})^{n-m}etr(-\mathbf{\Sigma}^{-1}\mathbf{A})d\mathbf{A}.
\label{Appendequa5}
\end{equation}
Because $\frac{d}{dn}[det(\mathbf{A})]^{n-m}=\ln(det(\mathbf{A}))det(\mathbf{A})^{n-m}$, (\ref{Appendequa5}) can be rewritten as
\begin{equation}
E[\ln(det(\mathbf{A}))]=\frac{1}{\tilde{\Gamma}_{m}(n)}\int_{\mathbf{A}=\mathbf{A}^{H}>0}det(\mathbf{A})^{n-m}etr(-\mathbf{A})d\mathbf{A},
\label{Appendequa6}
\end{equation}
using (\ref{Appendequa3}), (\ref{Appendequa6}) can be rewritten as 
\begin{equation}
E[\ln(\mathbf{A})]=\frac{\tilde{\Gamma}^{'}_{m}(n)}{\tilde{\Gamma}_{m}(n)}.
\label{Appendequa7}
\end{equation}
Based on (\ref{Appendequa2}), we have 
\begin{eqnarray}
\tilde{\Gamma}^{'}_{m}(n)=\pi^{\frac{m(m-1)}{2}}\prod_{i=1}^{m}\Gamma^{'}(\beta-i+1),
\end{eqnarray}
Thus we have 
\begin{equation}
E[\ln(det(\mathbf{A}))]=\frac{\tilde{\Gamma}^{'}_{m}(n)}{\tilde{\Gamma}_{m}(n)}=\prod_{i=1}^{m}\psi(n-i+1),
\label{Appendequa8}
\end{equation}
where $\psi$ denotes Digamma function.
\section{Appendix B}
If $x\sim Gamma(n, \theta)$, with shape parameter $k$ and scale parameter $\theta$, $x>0$, $\Gamma(k)$ denotes Gamma function, the density function of Gamma distribution is
\begin{equation}
f(x,k,\theta)=\frac{x^{k-1}e^{-x/\theta}}{\Gamma(k)\theta^{k}}.
\label{Appendequa9}
\end{equation}
Thus we have 
\begin{equation}
E[\ln(x)]=\frac{1}{\Gamma(k)}\int_{0}^{\infty}\ln(x)x^{k-1}e^{-x/\theta}\theta^{-k}dx,
\label{Appendequa10}
\end{equation}
define $z=x/\theta$ and since $\Gamma(k)=\int_{0}^{\infty}x^{k-1}e^{-x}dx$, (\ref{Appendequa10}) can be rewritten as
\begin{equation}
E[\ln(x)]=\ln(\theta)+\frac{1}{\Gamma(k)}\int_{0}^{\infty}\ln(z)z^{k-1}e^{-z}dz.
\label{Appendequa11}
\end{equation}
Because $\frac{d(z^{k-1})}{dk}=\ln(z)z^{k-1}$, (\ref{Appendequa11}) can be rewritten as
\begin{eqnarray}
\nonumber
E[\ln(z)]&=&\ln(\theta)+\frac{1}{\Gamma(k)}\frac{d}{dk}\int_{0}^{\infty}z^{k-1}e^{-z}dz\\
\nonumber
&=&\ln(\theta)+\frac{\Gamma^{'}(k)}{\Gamma(k)}\\
\nonumber
&=&\ln(\theta)+\psi(k),
\label{Appendequa12}
\end{eqnarray}
where $\psi(k)$ denotes Digamma function.

\end{appendices}





% that's all folks
\end{spacing}
\bibliographystyle{IEEEtran}
\bibliography{citation}
\end{document}