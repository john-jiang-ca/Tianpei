\resetdatestamp



\chapter{Complex Support Vector Preliminary Detector (CSVPD) for Large-Scale MIMO Uplink Systems}
Decomposition methods were proposed to solve this QP problem by decomposing it into sub QP problems and solving them iteratively\cite{platt1999fast}. Therefore, the computational intensive numerical methods can be avoided. Decomposition is performed by sub set selection solver, which refers to a set of algorithms that separate the optimization variables (Lagrange multipliers) into two sets S and N, S is the work set and N contains the remaining optimization variables. In each iteration, only the optimization variables in the work set is updated while keeping other optimization variables fixed. The Sequential Minimal Optimization (SMO) algorithm\cite{platt1999fast} is an extreme case of the decomposition solvers. An important issue of the sub set selection solvers is the selection of the work set. One strategy is to choose Karush-Kuhn-Tucker (KKT) condition violators, ensuring the final converge\cite{osuna1997improved}. The SMO algorithm restricts the size of the work set to two. A method to train SVM without offset was proposed In\cite{steinwart2011training}, with the comparable performance to the SVM with offset. A set of sequential single variable work set selection strategies, which require $O(n)$ searching time are proposed. The optimal double variable work set selection strategy, which performs exhaustively searching, however, requires $O(n^{2})$ searching time. The authors demonstrate that with the combination of two different proposed single variable work set selection strategies, convergence can be achieved by a iteration time that is as few as optimal double variable work set selection strategy.

The mathematical foundation of kernel based methods is RKHS which is defined in complex domain, however most of the practitioners are dealing with real data sets. In communication and signal processing area, the channel gains, signals, waveforms etc. are all represented in complex form. Recently, a pure complex SVR \& SVM based on complex kernel was proposed in\cite{bouboulis2013complex}, which can deal with the complex data set purely in complex domain. The results in\cite{bouboulis2013complex} demonstrate the better performance as well as reduced complexity comparing to simply split learning task into two real case by real kernels.  
Based on this work, we derive a complexity-performance controllable detector for large MIMO systems based on a dual channel complex SVR (CSVR). The detector can work in two parallel real SVR channels which can be solved independently. Moreover, only the real part of kernel matrix is needed in both channels. This means a large amount of computation cost saving can be achieved.
Based on the discrete time MIMO channel model, in our regression model, this CSVR-detector
is constructed without offset, Therefore, for each real SVR without offset, 
%in principle, only one variable is needed to be updated in each iteration, In our scheme, a sequential single variable selection strategy is proposed. By this strategy, two variables can be updated at each iteration, with much smaller searching time.
Two types of combined single optimization variable selection strategy are proposed based on the work in \cite{steinwart2011training}. The proposed combined single optimization variable selection strategy can approximate optimal double optimization variable selection strategy. The former one can achieve as few as iteration time while enjoy significant speed gain in each iteration.
\section{The Algorithm}