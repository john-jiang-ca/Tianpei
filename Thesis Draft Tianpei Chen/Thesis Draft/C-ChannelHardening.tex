\resetdatestamp


\chapter{Theoretical Analysis of Channel Hardening Phenomenon}\label{Channel Hardening}
\section{System Model}
Consider a uncoded complex large MIMO uplink spatial multiplexing (SM) system with $N_{t}$ users, where each is equipped with transmit antenna. The number of receive antennas at the Base Station (BS) is $N_{r}$, $N_{r}\geq N_{t}$. Typically large MIMO systems have hundreds of antennas at the BS, as shown in Fig {\ref{large MIMO uplink model}}.
  \begin{figure}[htb]
  \centering
  \def\svgwidth{\columnwidth}
  %\includegraphics[scale=•]{•}
  \input{largeMIMOuplink.pdf_tex}
  \caption{Large MIMO uplink system }
  \label{large MIMO uplink model}
  \end{figure}
    
Bit sequences, which are modulated to complex symbols, are transmitted by the users over a flat fading channel. The discrete time model of the system is given by:
\begin{equation}
\mathbf{y}=\mathbf{H}\mathbf{s}+\mathbf{n},   \label{discrete time MIMO system}
\end{equation}
where $\mathbf{y}\in\mathbb{C}^{N_{r}\times 1}$ is the received symbol vector, $\mathbf{s}\in \mathbb{C}^{N_{t}}$ is the transmitted symbol vector, with components that are mutually independent and taken from a finite signal constellation alphabet $\mathbb{O}$ (e.g. BPSK, 4-QAM, 16-QAM, 64-QAM), $|\mathbb{O}|=M$. The transmitted symbol vectors $\mathbf{s}\in \mathbb{O}^{N_{t}}$, satisfy $\mathbb{E}[\mathbf{s}\mathbf{s}^{H}]=\mathbf{I}_{N_t}E_{s}$, where $E_{s}$ denotes the symbol average energy, $\mathbb{E}[\cdot]$ denotes the expectation operation, $\mathbf{I}_{N_{t}}$ denotes identity matrix of size $N_{r}\times N_{t}$. Furthermore $\mathbf{H}\in \mathbb{C}^{N_{r}\times N_{t}}$ denotes the Rayleigh fading channel propagation matrix, each component is independent identically distributed (i.i.d) circularly symmetric complex Gaussian random variable with zero mean and unit variance. Finally, $\mathbf{n}\in \mathbb{C}^{N_{r}}$ is the additive white Gaussian noise (AWGN) vector with zero mean components and $\mathbb{E}[\mathbf{n}\mathbf{n}^{H}]=\mathbf{I}_{N_{r}}N_{0}$, where $N_{0}$ denotes the noise power spectrum density, and hence $\frac{E_{s}}{N_{0}}$ is the signal to noise ratio (SNR). 

The task of a MIMO detector is to estimate the transmit symbol vector $\mathbf{s}$, based on the knowledge of receive symbol vector $\mathbf{y}$ and channel matrix $\mathbf{H}$.