\resetdatestamp

\chapter{Low Complexity Near Optimal Hybrid Detector based on Genetic Algorithm}\label{Chapter GA}
In the previous chapter, we introduce two CSVPD aided local search based algorithms, and show the improvement CSVD-LAS -OPIC comparing to MMSE-LAS -OPIC.    
a current trend is the hybridization of of single solution based metaheuristics algorithms in population-based ones, working as intelligent subordinate heuristic operators. In this chapter, we introduce Hybrid Genetic Algorithms which utilizes OPIC as a genetic operator and accelerate the revolution process comparing to random revolution. Furthermore, instead of randomly generated initial population, the solution of the CSVPD is fed to Hybrid GA and the initial generation consists of the neighbour of the initial solution from CSVPD.  
\section{SVPD aided Hybrid Genetic Algorithm}
Genetic Algorithm (GA) is one of the population-based metaheuristic algorithms and can be viewed as a computational model of natural biological evolution process. GA works iteratively and in each iteration it handles a fixed size population of individuals. A number of intelligent genetic operators are used to update the population. Usually the individuals are a string of symbols that represent a candidate solution to the problems. GA has been applied in Code Division Multiple Access (CDMA) detection problem\cite{juntti1997genetic}\cite{ergun2000multiuser}.  

Figure (\ref{}) shows the flow chart of the proposed hybrid GA. In the beginning, the initial solution of CSVPD is fed to GA, a initial population of solutions are generated based on this initial solution, then in each iteration, each candidate solution is evaluated and selected by the fitness function. The elites are selected as parents which can generate offspring. The offsprings are generated by recombining a pair of parents, this process is called crossover. Then mutation operates to offsprings with a given possibility. OPIC is integrated in GA as an operator that can accelerate the process that GA approximate to optimal solution. The populations is finally reconstructed by the parents and their offsprings. The best solution is chosen based on the fitness function when the stopping criteria satisfied.
\subsection{Initialization}
Let $\mathbb{P}_{i}$ denotes the $i$th generation set consists of $|\mathbb{P}|$ symbol vectors, which is fixed in different generations. The estimated symbol vector from SVPD serves as the generation seed for the initial generation. The initial generation is generated by flipping a fix number of symbols to the nearest point in the symbol constellation alphabet randomly.

Let $\mathcal{F}$ denotes the number of symbols to be flipped. By flipping, a "$\mathcal{F}-coordinate-away$" neighbourhood symbol vector is generated, where "$\mathcal{F}-coordinate-away$" denotes the objective symbol vector has $\mathcal{F}$ coordinates different from the original symbol vector.

We now use a binary sequence generated by SVPD to illustrate how the flipping process works, let $b_{SVPD}\in \mathbb{B}^{N_{t}}, \mathbb{B}=[+1, -1], N_{t}=8$ denotes the the binary sequence generated by SVPD, 
\begin{equation}
b_{SVPD}=[1, -1, -1, 1, -1, 1, 1, -1],
\end{equation}  
$\mathcal{F}=1$, means there is one symbol difference between the original symbol vector and the objective symbol vector after flipping. Let $b_{1}, b_{2}, \ldots b_{|\mathbb{P}|-1}$ denotes the individuals in the initial generation which are generated by perturbing seed vector $b_{SVPD}$. Fig.\ref{random flipping} shows how this random flipping scheme works.
\begin{figure}
\centering
\def\svgwidth{\columnwidth}
\input{randomFlipping.pdf_tex}
\caption{Random Flipping with $\mathcal{F}=1$}
\label{random fipping}
\end{figure}
The initial generation is $\mathbb{P}=[b_{SVPD}, b_{1}, b_{2}, \ldots b_{|\mathbb{P}|-1}$.
\subsection{Selection}
The individuals in each generation are evaluated by the fitness function. In LS-MIMO detection problem. Euclidean distance is used as the fitness function, given by 
\begin{equation}
f(\mathbf{s_{i}})=||\mathbf{y}-\mathbf{H}\mathbf{s}_{i}||^{2}, 
\label{fitness function}
\end{equation}
where $\mathbf{s}_{i}$ denotes the $i$th individual symbol vector in the generation, the individual with the smallest fitness function value is the fittest candidate (elite) in the generation.
In each generation, the individuals with small $f(\mathbf{s}_{i})$ are chosen as the parents to generate offsprings, the offsprings are generated by recombining the parent symbol vectors, this process is recombination or crossover. In crossover process, crossover point is randomly chosen in parent symbol vectors and the subpart of a pair of parent symbol vectors up to that crossover point are interchanged. A simple cased of this process is shown in Fig.\ref{crossover}.

\begin{figure}
\centering
\def\svgwidth{\columnwidth}
\input{crossover.pdf_tex}
\caption{Crossover operation to a pair of parent binary sequence}
\label{crossover}
\end{figure}

\subsection{Mutation}
The GA in operates in a sub searching space, which is initially, determined by the $\mathcal{F}$ neighbourhood of the seed symbol vector. The major difficulties in the revolution process in the convergence to the local minima of the sub searching space. Mutation is a strategy that helps GA to escape from local minima. Mutations operates by randomly perturbation of a small number of coordinates in offspring symbol vectors with a small possibility defined by the users. Fig.\ref{mutation} shows how mutation works. 
\iffalse
\begin{figure}
\centering
\def\svgwidth{\columnwidth}
\input{mutation.pdf_tex}
\caption{Mutation operation to one bit in binary sequence}
\label{mutaion}
\end{figure}
\fi


\begin{figure}
\centering
\def\svgwidth{\columnwidth}
\input{mutation.pdf_tex}
\caption{Mutation operation to one bit in binary sequence}
\label{mutation}
\end{figure}


\subsection{OPIC} 
\subsection{Stopping Criteria}
 