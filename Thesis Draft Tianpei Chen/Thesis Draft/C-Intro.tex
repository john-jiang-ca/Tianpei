\resetdatestamp

%\chapter{Thesis Organization}

%A thesis should present results in a scholarly fashion.
%The following discusses the organization of a thesis, such as would be
 %appropriate to presenting research results pertaining to Electrical \&
% Computer Engineering.

%\section{Scope of a Thesis}

%The terms of reference differ for a Master's Thesis and a Doctoral Thesis.
%For the Master's Thesis, the Faculty of Graduate Studies and Research at
%McGill University \cite{McGillTG:P1994} gives the following guidelines.

%The terms of reference state that ``In most disciplines, Master's theses will
% not exceed 100 pages.''


%==========  main content of thesis
\chapter{Introduction} \label{introduction}
\section{Large MIMO system}

  One of the biggest challenges the researchers and industry practitioners are facing in wireless communication area is how to bridge the sharp gap between increasing demand of high speed communication of rich multimedia information with high level Quality of Service (QoS) requirements and the limited radio frequency spectrum over a complex space-time varying environment. The promising technology for solving this problem, Multiple Input Multiple Output (MIMO) technology has been of immense research interest over the last several tens of years is incorporated into the emerging wireless broadband standard like 802.11ac\cite{IEEE802.11ac} and long-term evolution (LTE)\cite{3GLTE}.  The core idea of MIMO system is to use multiple antennas at both transmitting and receiving end, so that multiplexing gain (multiple parallel spatial data pipelines that can improve spectrum efficiency) and diversity gain (better reliability of communication link) are obtained by exploiting the spatial domain\cite{oestges2010mimo}. Large-Scale MIMO system (LS-MIMO) is an upgraded version of conventional MIMO technology employing hundreds of low power low price antennas at base station (BS), that serves several tens of terminals simultaneously. This technology can achieve full potential of conventional MIMO systems including significant link reliability and throughput benefits. Nonetheless, the LS-MIMO systems have been shown to enjoy some distinct advantages
that are not available in small-scale MIMO systems such as additional power efficiency as well as system robustness to both unintended man-made interference and intentional jamming.\cite{rusek2013scaling}\cite{larsson2014massive}. 
\section{Large MIMO detections}
%Linear detectors (LD) like minimum mean square error (MMSE) and zero forcing (ZF) along with their sequential interference cancellation with optimized ordering (OSIC) counterparts\cite{wolniansky1998v}, which have good performance for low loading factor in massive MIMO system (the number of receive antennas is much larger than the number of transmit antennas)\cite{hoydis2013massive}. 
 The price paid for LS-MIMO system is the increased complexities for signal processing at both transmitting and receiving ends. The uplink detector is one of the key components in a large MIMO system. With orders magnitude more antennas at the BS, benefits and challenges coexist in designing of detection algorithms for the uplink communication of large MIMO systems. On one hand, a large number of receive antennas provide potential of large diversity gains, on the other hand, complexities of the algorithms become crucial to make the system practical. 
  
Vertical Bell Laboratories Layered Space-Time (V-BLAST) architecture for MIMO system can achieve high spectrum efficiency by spatial multiplexing (SM), that is, each transmit antenna transmits independent symbol streams. However the optimal maximum likelihood detector (MLD) for V-BLAST systems that perform exhaustive search over the transmit symbol vector space has a complexity that increases exponentially with the number of transmitted antennas, which is prohibitive for practical applications. 

In order to alleviate this problem, linear detectors (LD) such as zero-forcing (ZF) and minimum mean square error (MMSE) aided by successive interference cancellation with optimal ordering (ZF-OSIC, MMSE-OSIC) are exploited in V-BLAST architecture\cite{wolniansky1998v}\cite{foschini1999simplified}\cite{benesty2003fast}, although ZF-OSIC and MMSE-OSIC can provide significant improvement comparing to their LD counterparts, a common drawback of SIC aided LD is the error propagation effect, which can not be eliminated by ordering. That results in inferior performances comparing to MLD\cite{loyka2004performance}\cite{prasad2004analysis}\cite{jiang2005asymptotic}.
   
Sphere Decoder (SD)\cite{damen2003maximum} is the most prominent algorithm that utilizes the lattice structure of MIMO systems, which can achieve optimal performance with relatively much lower complexity comparing to MLD. However, SD has two major shortages that make it problematic to be integrated into a practical systems. The first shortage is SD has various complexities under different signal to noise ratios (SNR), while a constant processing data rate is required for hardware. The second shortage is the complexity of SD still has a lower bound that increases exponentially with the number of transmit antennas and the order of modulation scheme\cite{jalden2005complexity}.
The fixed complexity sphere decoder (FCSD)\cite{barbero2008fixing} makes it possible to achieve near optimal performance with a fixed complexity under different values of SNR. The FCSD inherits the principle of list based searching algorithms, which first generate a list of candidate symbol vectors and then the best candidate is chosen as the solution. The other sub optimal detectors belong to this class include Generalized Parallel Interference Cancellation (GPIC)\cite{luo2008generalized} and Selection based MMSE-OSIC (sel-MMSE-OSIC)\cite{radji2009interference}. However, all these list based searching algorithms have the same shortage - their complexities increase exponentially with the number of transmit antennas and the order of modulation scheme\cite{radji2009interference}. Therefore, such algorithms are prohibitive when it comes to a large number of antennas or a high order modulation scheme, for example in IEEE 802.11ac standard\cite{IEEE802.11ac}, the modulation scheme is 256QAM. 
 
%Linear detectors (LD) like minimum mean square error (MMSE) and zero forcing (ZF) along with their sequential interference cancellation with optimized ordering (OSIC) counterparts\cite{wolniansky1998v}, which have good performance for low loading factor in massive MIMO system (the number of receive antennas is much larger than the number of transmit antennas)\cite{hoydis2013massive}. 
The key challenge in designing Large-Scale MIMO (LS-MIMO) detectors is to reduce complexities while maintaining high performances. Given the redundant diversity gain potential provided by LS-MIMO\cite{oestges2010mimo} that can only be achieved at very high Signal to Noise Ratio (SNR), one may consider to sacrifice the redundant diversity gains in order to reduce complexities. Furthermore, the ML criterion of MIMO detection stated in (\ref{MLD}) indicates MIMO detection can be naturally classified as a Combinatorial Optimization (CO) problem, that consist of the search for the optimal solution in a discrete and finite set. 

Based on the aforementioned situations, the metaheuristics algorithmic frameworks are of intense research interest in LS-MIMO detection in recent years. Metaheuristics refer to the high level algorithmic strategies that guide subordinate heuristics iteratively by combining different intelligent learning operations. Metaheuristics can explore and exploit the searching space efficiently and find near optimal solutions without costing high complexities. The class of algorithms includes but not restrict to Iterative Local Search (ILS), Tabu Search (TS), Simulated Annealing (SA), Genetic Algorithms (GA) and Ant Colony Optimization (ACO).

Therefore besides the above detection algorithms designed for conventional MIMO systems, in the last several years, a variety of metaheuristic based local search algorithms invoked from machine learning field\cite{chockalingam2010low} have been proposed for LS-MIMO systems. These algorithms have complexities that are comparable or slightly higher comparing to MMSE detector and near-optimal performance. Such algorithms include likelihood ascend searching (LAS) algorithm and variants\cite{vardhan2008low}\cite{cerato2009hardware}\cite{li2010multiple} %An theoretical analysis of upper bound of bit error rate (BER) and lower bound of on asymptotic multiuser efficiency for the LAS detector was also presented\cite{sunfamily}.
and Reactive Tabu search (RTS) algorithms and variants
%which have superior performances compared to LAS detectors because of the efficient local minima exit strategy 
(e.g. Layered Tabu search (LTS)\cite{srinidhi2011layered}, Random Restart Reactive Tabu search (R3TS)\cite{datta2010random}). Additionally, there are other algorithms proposed for large MIMO systems including Message passing technique based algorithms (e.g. Belief propagation (BP) detectors based on graphic model and Gaussian Approximation (GA)\cite{som2011low}\cite{som2010improved}\cite{narasimhan2014channel}\cite{goldberger2011mimo}), Probabilistic Data Association (PDA) based algorithms \cite{mohammed2009low}, Monte Carlo sampling technique based algorithms (e.g. Multiple Restart- Mixed Gibbs Sampling (MR-MGS) algorithm\cite{datta2013novel}) and Element based Lattice Reduction (LR) aided algorithms\cite{zhou2013element}.

 Considering MIMO detection from a Combinatorial Optimization (CO) problem viewpoint, as powerful tool for solving CO problems, methoheuristic algorithms\cite{blum2003metaheuristics} are good choices for designing large MIMO detectors, driven by demand of achieving acceptable performance with significantly lower computational complexity. Besides the metaheuristic algorithms that based on local search strategies which use trajectory methods based on single solution, another class of metaheuristic algorithms is defined as population based. The population based metaheuristic algorithms deal with a population of candidate solutions. Intrinsically the population based metaheuristic algorithms can provide wider and more efficient exploration of search space. The major population based metaheuristic algorithms includes Evolutionary Computation (EC) and Ant Colony Optimization (ACO). 
 
 Genetic algorithm (GA) is one kind of EC algorithms, which are designed for CO problems. GA mimics the natural evolution process of a population and is powerful tool in searching a solution that is close enough to global optimum\cite{Goldberg:1989:GAS:534133}.
 
 \section{Support Vector Regression}
 
 Firmly grounded in framework of statistical learning theory or VC (Vapnik--Chervonenkis) theory, the Support Vector (SV) technique has become a powerful tool to solve real world supervised learning problems such as classification, regression and prediction. The SV method is a nonlinear generalization of Generalized Portrait algorithm developed by Vapnik in 1960s\cite{vapnik1963pattern}\cite{vapnik1964note}, which can provide good generalization performance\cite{scholkopf2002learning}.

Interest in SV algorithms boosted since 1990s, promoted by the works of Vapnik and co-workers at AT\& T Bell laboratory\cite{boser1992training}\cite{guyon1993automatic}\cite{vapnik2013nature}\cite{cortes1995support}\cite{scholkopf1996incorporating}\cite{vapnik1996support}.
Moreover, the kernel based methods\cite{scholkopf2002learning} were proposed in order to extending the SV algorithms to nonlinear learning cases. The input data samples are mapped into some high dimensional feature space (also called Reproducing Kernel Hilbert Space (RKHS)) and then linear tools are applied to the feature mappings of the input data samples. This is equivalent to transforming the nonlinear learning tasks in the original space into the linear learning tasks in high dimensional feature space. The mathematical notion underlying kernel based methods is that of RKHS\cite{scholkopf2002learning}, in which the inner products of the feature mappings can be simply replaced by the computationally economical kernel functions. Because SV algorithms only deal with the inner products of the the feature mappings. Therefore, by kernel based methods, it is sufficient to use the specific kernel functions based on the RKHS discarding the actual structure of the feature space.   
 
Based on the principle of structural risk minimization\cite{vapnik1982estimation}, the $\epsilon$-SVR\cite{vapnik2013nature}\cite{smola2004tutorial} solves an original constraint optimization problem (primal objective problem) by transforming it into a Lagrange dual form (dual objective problem), which is a Quadratic Programming (QP) problem. Efficient methods for training SV algorithms which are based on large scale data sets were proposed, which is called decomposition. Decomposition process is performed by decomposing a large QP problem into a sequence of sub QP problems and solve them in an iterative manner\cite{osuna1997training}\cite{osuna1997improved}, Sequential Minimal Optimization (SMO) algorithm\cite{platt1999fast} is one of well known representatives of decomposition methods. 

Complex valued signal arises in signal processing and digital communication areas etc. Therefore, developing signal processing algorithms which are suitable to be directly applied to complex valued systems is typically natural and concise choice. Furthermore, for MIMO systems, although one can transform the complex valued system model into an equivalent real valued system model, the detection algorithms built based on the complex valued model is more preferable due to the flexibility for signal constellation choice and hardware implementations\cite{fiftyYang2015}. In recent years, a mathematical framework for pure complex valued SV algorithms was developed which can deal with complex valued tasks in signal processing, digital communication and related areas in an elegant and computational efficient manner\cite{bouboulis2013complex}.  


\section{Thesis Contribution}


\section{Thesis Outline}