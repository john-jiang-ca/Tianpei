\documentclass [12pt,letterpaper]{report}
%\documentclass[12pt, draftclsnofoot, onecolumn]{IEEEtran}
% Standard packages
%\usepackage{amsmath}		% Extra math definitions
\usepackage{graphics}		% PostScript figures
\usepackage{setspace}		% 1.5 spacing
\usepackage{longtable}          % Tables spanning pages
\usepackage[latin1]{inputenc}
\usepackage{amsfonts}
\usepackage{amssymb}
\usepackage{amsthm}
\usepackage{fullpage}
\usepackage{setspace}
\usepackage{graphicx}
%\usepackage[pdftex]{graphicx}
\usepackage{psfrag}
\usepackage{color}
\usepackage{epsfig}
%\usepackage{appendix}
%\usepackage{caption}
\usepackage{cite}
\usepackage{ifpdf}
\usepackage{algorithm}
\usepackage{array}
\usepackage{stfloats}
\usepackage{url}
\usepackage{fixltx2e}
\usepackage{setspace} 
\usepackage{diagbox}
\usepackage{subfigure}
\usepackage{algpseudocode}
\usepackage{multirow}
\usepackage{calc}
% Some very useful LaTeX packages include:
% (uncomment the ones you want to load)


% *** MISC UTILITY PACKAGES ***
%
\usepackage{ifpdf}
% Heiko Oberdiek's ifpdf.sty is very useful if you need conditional
% compilation based on whether the output is pdf or dvi.
% usage:
% \ifpdf
%   % pdf code
% \else
%   % dvi code
% \fi
% The latest version of ifpdf.sty can be obtained from:
% http://www.ctan.org/tex-archive/macros/latex/contrib/oberdiek/
% Also, note that IEEEtran.cls V1.7 and later provides a builtin
% \ifCLASSINFOpdf conditional that works the same way.
% When switching from latex to pdflatex and vice-versa, the compiler may
% have to be run twice to clear warning/error messages.






% *** CITATION PACKAGES ***
%
\usepackage{cite}
% cite.sty was written by Donald Arseneau
% V1.6 and later of IEEEtran pre-defines the format of the cite.sty package
% \cite{} output to follow that of IEEE. Loading the cite package will
% result in citation numbers being automatically sorted and properly
% "compressed/ranged". e.g., [1], [9], [2], [7], [5], [6] without using
% cite.sty will become [1], [2], [5]--[7], [9] using cite.sty. cite.sty's
% \cite will automatically add leading space, if needed. Use cite.sty's
% noadjust option (cite.sty V3.8 and later) if you want to turn this off
% such as if a citation ever needs to be enclosed in parenthesis.
% cite.sty is already installed on most LaTeX systems. Be sure and use
% version 5.0 (2009-03-20) and later if using hyperref.sty.
% The latest version can be obtained at:
% http://www.ctan.org/tex-archive/macros/latex/contrib/cite/
% The documentation is contained in the cite.sty file itself.






% *** GRAPHICS RELATED PACKAGES ***
%
%\ifCLASSINFOpdf
  % \usepackage[pdftex]{graphicx}
  % declare the path(s) where your graphic files are
  % \graphicspath{{../pdf/}{../jpeg/}}
  % and their extensions so you won't have to specify these with
  % every instance of \includegraphics
  % \DeclareGraphicsExtensions{.pdf,.jpeg,.png}
%\else
  % or other class option (dvipsone, dvipdf, if not using dvips). graphicx
  % will default to the driver specified in the system graphics.cfg if no
  % driver is specified.
  % \usepackage[dvips]{graphicx}
  % declare the path(s) where your graphic files are
  % \graphicspath{{../eps/}}
  % and their extensions so you won't have to specify these with
  % every instance of \includegraphics
  % \DeclareGraphicsExtensions{.eps}
%\fi
% graphicx was written by David Carlisle and Sebastian Rahtz. It is
% required if you want graphics, photos, etc. graphicx.sty is already
% installed on most LaTeX systems. The latest version and documentation
% can be obtained at: 
% http://www.ctan.org/tex-archive/macros/latex/required/graphics/
% Another good source of documentation is "Using Imported Graphics in
% LaTeX2e" by Keith Reckdahl which can be found at:
% http://www.ctan.org/tex-archive/info/epslatex/
%
% latex, and pdflatex in dvi mode, support graphics in encapsulated
% postscript (.eps) format. pdflatex in pdf mode supports graphics
% in .pdf, .jpeg, .png and .mps (metapost) formats. Users should ensure
% that all non-photo figures use a vector format (.eps, .pdf, .mps) and
% not a bitmapped formats (.jpeg, .png). IEEE frowns on bitmapped formats
% which can result in "jaggedy"/blurry rendering of lines and letters as
% well as large increases in file sizes.
%
% You can find documentation about the pdfTeX application at:
% http://www.tug.org/applications/pdftex





% *** MATH PACKAGES ***
%
\usepackage[cmex10]{amsmath}
% A popular package from the American Mathematical Society that provides
% many useful and powerful commands for dealing with mathematics. If using
% it, be sure to load this package with the cmex10 option to ensure that
% only type 1 fonts will utilized at all point sizes. Without this option,
% it is possible that some math symbols, particularly those within
% footnotes, will be rendered in bitmap form which will result in a
% document that can not be IEEE Xplore compliant!
%
% Also, note that the amsmath package sets \interdisplaylinepenalty to 10000
% thus preventing page breaks from occurring within multiline equations. Use:
%\interdisplaylinepenalty=2500
% after loading amsmath to restore such page breaks as IEEEtran.cls normally
% does. amsmath.sty is already installed on most LaTeX systems. The latest
% version and documentation can be obtained at:
% http://www.ctan.org/tex-archive/macros/latex/required/amslatex/math/





% *** SPECIALIZED LIST PACKAGES ***
%
%\usepackage{algorithmic}
% algorithmic.sty was written by Peter Williams and Rogerio Brito.
% This package provides an algorithmic environment fo describing algorithms.
% You can use the algorithmic environment in-text or within a figure
% environment to provide for a floating algorithm. Do NOT use the algorithm
% floating environment provided by algorithm.sty (by the same authors) or
% algorithm2e.sty (by Christophe Fiorio) as IEEE does not use dedicated
% algorithm float types and packages that provide these will not provide
% correct IEEE style captions. The latest version and documentation of
% algorithmic.sty can be obtained at:
% http://www.ctan.org/tex-archive/macros/latex/contrib/algorithms/
% There is also a support site at:
% http://algorithms.berlios.de/index.html
% Also of interest may be the (relatively newer and more customizable)
% algorithmicx.sty package by Szasz Janos:
% http://www.ctan.org/tex-archive/macros/latex/contrib/algorithmicx/




% *** ALIGNMENT PACKAGES ***
%
\usepackage{array}
% Frank Mittelbach's and David Carlisle's array.sty patches and improves
% the standard LaTeX2e array and tabular environments to provide better
% appearance and additional user controls. As the default LaTeX2e table
% generation code is lacking to the point of almost being broken with
% respect to the quality of the end results, all users are strongly
% advised to use an enhanced (at the very least that provided by array.sty)
% set of table tools. array.sty is already installed on most systems. The
% latest version and documentation can be obtained at:
% http://www.ctan.org/tex-archive/macros/latex/required/tools/


% IEEEtran contains the IEEEeqnarray family of commands that can be used to
% generate multiline equations as well as matrices, tables, etc., of high
% quality.




% *** SUBFIGURE PACKAGES ***
%\ifCLASSOPTIONcompsoc
 % \usepackage[caption=false,font=normalsize,labelfont=sf,textfont=sf]{subfig}
%\else
 % \usepackage[caption=false,font=footnotesize]{subfig}
%\fi
% subfig.sty, written by Steven Douglas Cochran, is the modern replacement
% for subfigure.sty, the latter of which is no longer maintained and is
% incompatible with some LaTeX packages including fixltx2e. However,
% subfig.sty requires and automatically loads Axel Sommerfeldt's caption.sty
% which will override IEEEtran.cls' handling of captions and this will result
% in non-IEEE style figure/table captions. To prevent this problem, be sure
% and invoke subfig.sty's "caption=false" package option (available since
% subfig.sty version 1.3, 2005/06/28) as this is will preserve IEEEtran.cls
% handling of captions.
% Note that the Computer Society format requires a larger sans serif font
% than the serif footnote size font used in traditional IEEE formatting
% and thus the need to invoke different subfig.sty package options depending
% on whether compsoc mode has been enabled.
%
% The latest version and documentation of subfig.sty can be obtained at:
% http://www.ctan.org/tex-archive/macros/latex/contrib/subfig/




% *** FLOAT PACKAGES ***
%
\usepackage{fixltx2e}
% fixltx2e, the successor to the earlier fix2col.sty, was written by
% Frank Mittelbach and David Carlisle. This package corrects a few problems
% in the LaTeX2e kernel, the most notable of which is that in current
% LaTeX2e releases, the ordering of single and double column floats is not
% guaranteed to be preserved. Thus, an unpatched LaTeX2e can allow a
% single column figure to be placed prior to an earlier double column
% figure. The latest version and documentation can be found at:
% http://www.ctan.org/tex-archive/macros/latex/base/


\usepackage{stfloats}
% stfloats.sty was written by Sigitas Tolusis. This package gives LaTeX2e
% the ability to do double column floats at the bottom of the page as well
% as the top. (e.g., "\begin{figure*}[!b]" is not normally possible in
% LaTeX2e). It also provides a command:
%\fnbelowfloat
% to enable the placement of footnotes below bottom floats (the standard
% LaTeX2e kernel puts them above bottom floats). This is an invasive package
% which rewrites many portions of the LaTeX2e float routines. It may not work
% with other packages that modify the LaTeX2e float routines. The latest
% version and documentation can be obtained at:
% http://www.ctan.org/tex-archive/macros/latex/contrib/sttools/
% Do not use the stfloats baselinefloat ability as IEEE does not allow
% \baselineskip to stretch. Authors submitting work to the IEEE should note
% that IEEE rarely uses double column equations and that authors should try
% to avoid such use. Do not be tempted to use the cuted.sty or midfloat.sty
% packages (also by Sigitas Tolusis) as IEEE does not format its papers in
% such ways.
% Do not attempt to use stfloats with fixltx2e as they are incompatible.
% Instead, use Morten Hogholm'a dblfloatfix which combines the features
% of both fixltx2e and stfloats:
%
% \usepackage{dblfloatfix}
% The latest version can be found at:
% http://www.ctan.org/tex-archive/macros/latex/contrib/dblfloatfix/




%\ifCLASSOPTIONcaptionsoff
 % \usepackage[nomarkers]{endfloat}
 %\let\MYoriglatexcaption\caption
% \renewcommand{\caption}[2][\relax]{\MYoriglatexcaption[#2]{#2}}
%\fi
% endfloat.sty was written by James Darrell McCauley, Jeff Goldberg and 
% Axel Sommerfeldt. This package may be useful when used in conjunction with 
% IEEEtran.cls'  captionsoff option. Some IEEE journals/societies require that
% submissions have lists of figures/tables at the end of the paper and that
% figures/tables without any captions are placed on a page by themselves at
% the end of the document. If needed, the draftcls IEEEtran class option or
% \CLASSINPUTbaselinestretch interface can be used to increase the line
% spacing as well. Be sure and use the nomarkers option of endfloat to
% prevent endfloat from "marking" where the figures would have been placed
% in the text. The two hack lines of code above are a slight modification of
% that suggested by in the endfloat docs (section 8.4.1) to ensure that
% the full captions always appear in the list of figures/tables - even if
% the user used the short optional argument of \caption[]{}.
% IEEE papers do not typically make use of \caption[]'s optional argument,
% so this should not be an issue. A similar trick can be used to disable
% captions of packages such as subfig.sty that lack options to turn off
% the subcaptions:
% For subfig.sty:
% \let\MYorigsubfloat\subfloat
% \renewcommand{\subfloat}[2][\relax]{\MYorigsubfloat[]{#2}}
% However, the above trick will not work if both optional arguments of
% the \subfloat command are used. Furthermore, there needs to be a
% description of each subfigure *somewhere* and endfloat does not add
% subfigure captions to its list of figures. Thus, the best approach is to
% avoid the use of subfigure captions (many IEEE journals avoid them anyway)
% and instead reference/explain all the subfigures within the main caption.
% The latest version of endfloat.sty and its documentation can obtained at:
% http://www.ctan.org/tex-archive/macros/latex/contrib/endfloat/
%
% The IEEEtran \ifCLASSOPTIONcaptionsoff conditional can also be used
% later in the document, say, to conditionally put the References on a 
% page by themselves.




% *** PDF, URL AND HYPERLINK PACKAGES ***
%
\usepackage{url}
% url.sty was written by Donald Arseneau. It provides better support for
% handling and breaking URLs. url.sty is already installed on most LaTeX
% systems. The latest version and documentation can be obtained at:
% http://www.ctan.org/tex-archive/macros/latex/contrib/url/
% Basically, \url{my_url_here}.




% *** Do not adjust lengths that control margins, column widths, etc. ***
% *** Do not use packages that alter fonts (such as pslatex).         ***
% There should be no need to do such things with IEEEtran.cls V1.6 and later.
% (Unless specifically asked to do so by the journal or conference you plan
% to submit to, of course. )


% correct bad hyphenation here
%\hyphenation{op-tical net-works semi-conduc-tor}
% Custom packages
\usepackage[first]{datestamp}	% Datestamp on first page of each chapter
\usepackage[fancyhdr]{McECEThesis}	% Thesis style
\usepackage{McGillLogo}		% McGill University crest

% $Id: ThesisEx.tex,v 1.1 2005/06/09 12:48:46 kabal Exp $

\usepackage{color}
\def\headrulehook{\color{red}}		% Color the header rule




%===== page layout
% Define the side margins for a right-side page
\insidemargin = 1.3in
\outsidemargin = 0.8in

% Above margin is space above the header
% Below margin is space below footer
\abovemargin = 1.1in
\belowmargin = 0.75in

%========= Document start

\begin {document}

%===== Title page

\title{ Low Complexity Near Optimal Hybrid Detectors for Large-Scale MIMO Uplink Systems Based on Complex Support Vector Regression}
\author{Tianpei Chen}
\date{\Month\ \number\year}
\organization{%
  \\[0.2in]
  \McGillCrest {!}{1in}\\	% McGill University crest
  \\[0.1in]
  Department of Electrical \& Computer Engineering\\
  McGill University\\
  Montreal, Quebec, Canada}
\note{%
  {\color{red} \hrule height 0.4ex}
  \vskip 3ex
  A thesis submitted to McGill University in partial fulfillment of the
  requirements for the degree of Master of Engineering.
  \vskip 3ex
  \copyright\ \the\year\ Tianpei Chen
}

\maketitle

%===== Justification, spacing for the main text
\raggedbottom
\onehalfspacing
\pagenumbering{roman}

%===== Abstract, Sommaire & Acknowledgments
\section*{\centering Abstract}

This report describes the use of \LaTeX{} to format a thesis.
A number of topics are covered: content and organization of the thesis,
 \LaTeX{} macros for controlling the thesis layout, formatting mathematical
 expressions, generating bibliographic references, importing figures and
 graphs, generating graphs in {\small MATLAB}, and formatting tables.
The \LaTeX{} macros used to format a thesis (and this document) are
 described.

\newpage

\section*{\centering Acknowledgments}

Thesis regulations require that contributions by others in the collection of
 materials and data, the design and construction of apparatus, the performance
 of experiments, the analysis of data, and the preparation of the thesis be
 acknowledged.
\newpage

     

%========== Tables of contents, figures, tables
\tableofcontents
\listoffigures
\listoftables

\newpage
\chapter*{List of Acronyms}\markright{List of Terms}

\begin{longtable}{ll}
  16-QAM   &  16-point Quadrature Amplitude Modulation\\
  3GPP     &  Third Generation Partnership Project\\
  3GPP2    &  Third Generation Partnership Project 2\\
  64-QAM   &  64-point Quadrature Amplitude Modulation\\
  ADSL     &  Asymmetric Digital Subscriber Line\\
  ARQ      &  Automatic Repeat Request\\
  WPAN     &  Wireless Personal Area Network
\end{longtable}

\cleardoublepage
\pagenumbering{arabic}

%========== Chapters
\typeout{}
\resetdatestamp

%\chapter{Thesis Organization}

%A thesis should present results in a scholarly fashion.
%The following discusses the organization of a thesis, such as would be
 %appropriate to presenting research results pertaining to Electrical \&
% Computer Engineering.

%\section{Scope of a Thesis}

%The terms of reference differ for a Master's Thesis and a Doctoral Thesis.
%For the Master's Thesis, the Faculty of Graduate Studies and Research at
%McGill University \cite{McGillTG:P1994} gives the following guidelines.

%The terms of reference state that ``In most disciplines, Master's theses will
% not exceed 100 pages.''


%==========  main content of thesis
\chapter{Introduction}
\section{Large MIMO system}

  One of the biggest challenges the researchers and industry practitioners are facing in wireless communication area is how to bridge the sharp gap between increasing demand of high speed communication of rich multimedia information with high level Quality of Service (QoS) requirements and the limited radio frequency spectrum over a complex space-time varying environment. The promising technology for solving this problem, Multiple Input Multiple Output (MIMO) technology has been of immense research interest over the last several tens of years is incorporated into the emerging wireless broadband standard like 802.11ac\cite{IEEE802.11ac} and long-term evolution (LTE)\cite{3GLTE}.  The core idea of MIMO system is to use multiple antennas at both transmitting and receiving end, so that multiplexing gain (multiple parallel spatial data pipelines that can improve spectrum efficiency) and diversity gain (better reliability of communication link) are obtained by exploiting the spatial domain. Large MIMO (also called Massive MIMO) is an upgraded version of conventional MIMO technology employing hundreds of low power low price antennas at base station (BS), that serves several tens of terminals simultaneously. This technology can achieve full potential of conventional MIMO system while providing additional power efficiency as well as system robustness to both unintended man-made interference and intentional jamming.\cite{rusek2013scaling}\cite{larsson2014massive}. 
\section{Large MIMO detections}
%Linear detectors (LD) like minimum mean square error (MMSE) and zero forcing (ZF) along with their sequential interference cancellation with optimized ordering (OSIC) counterparts\cite{wolniansky1998v}, which have good performance for low loading factor in massive MIMO system (the number of receive antennas is much larger than the number of transmit antennas)\cite{hoydis2013massive}. 
 The price paid for large MIMO system is the increased complexities for signal processing at both transmitting and receiving ends. The uplink detector is one of the key components in a large MIMO system. With orders magnitude more antennas at the BS, benefits and challenges coexist in designing of detection algorithms for the uplink communication of large MIMO systems. On one hand, a large number of receive antennas provide potential of large diversity gains, on the other hand, complexities of the algorithms become crucial to make the system practical. 
  
Vertical Bell Laboratories Layered Space-Time (V-BLAST) architecture for MIMO system can achieve high spectrum efficiency by spatial multiplexing (SM), that is, each transmit antenna transmits independent symbol streams. However the optimal maximum likelihood detector (MLD) for V-BLAST systems that perform exhaustive search over the transmit symbol vector space has a complexity that increases exponentially with the number of transmitted antennas, which is prohibitive for practical applications. 

In order to alleviate this problem, linear detectors (LD) such as zero-forcing (ZF) and minimum mean square error (MMSE) aided by successive interference cancellation with optimal ordering (ZF-OSIC, MMSE-OSIC) are exploited in V-BLAST architecture\cite{wolniansky1998v}\cite{foschini1999simplified}\cite{benesty2003fast}, although ZF-OSIC and MMSE-OSIC can provide significant improvement comparing to their LD counterparts, a common drawback of SIC aided LD is the error propagation effect, which can not be eliminated by ordering. That results in inferior performances comparing to MLD\cite{loyka2004performance}\cite{prasad2004analysis}\cite{jiang2005asymptotic}.
   
Sphere Decoder (SD)\cite{damen2003maximum} is the most prominent algorithm that utilizes the lattice structure of MIMO systems, which can achieve optimal performance with relatively much lower complexity comparing to MLD. However, SD has two major shortages that make it problematic to be integrated into a practical systems. The first shortage is SD has various complexities under different signal to noise ratios (SNR), while a constant processing data rate is required for hardware. The second shortage is the complexity of SD still has a lower bound that increases exponentially with the number of transmit antennas and the order of modulation scheme\cite{jalden2005complexity}.
The fixed complexity sphere decoder (FCSD)\cite{barbero2008fixing} makes it possible to achieve near optimal performance with a fixed complexity under different values of SNR. The FCSD inherits the principle of list based searching algorithms, which first generate a list of candidate symbol vectors and then the best candidate is chosen as the solution. The other sub optimal detectors belong to this class include Generalized Parallel Interference Cancellation (GPIC)\cite{luo2008generalized} and Selection based MMSE-OSIC (sel-MMSE-OSIC)\cite{radji2009interference}. However, all these list based searching algorithms have the same shortage - their complexities increase exponentially with the number of transmit antennas and the order of modulation scheme\cite{radji2009interference}. Therefore, such algorithms are prohibitive when it comes to a large number of antennas or a high order modulation scheme, for example in IEEE 802.11ac standard\cite{IEEE802.11ac}, the modulation scheme is 256QAM. 
 
%Linear detectors (LD) like minimum mean square error (MMSE) and zero forcing (ZF) along with their sequential interference cancellation with optimized ordering (OSIC) counterparts\cite{wolniansky1998v}, which have good performance for low loading factor in massive MIMO system (the number of receive antennas is much larger than the number of transmit antennas)\cite{hoydis2013massive}. 
Besides the above detection algorithms designed for conventional MIMO systems, in the last several years, a variety of metaheuristic based local search algorithms invoked from machine learning field\cite{chockalingam2010low} have been proposed for large MIMO systems. These algorithms have complexities that are comparable or slightly higher comparing to MMSE detector and near-optimal performance. Such algorithms include likelihood ascend searching (LAS) algorithm and variants\cite{vardhan2008low}\cite{cerato2009hardware}\cite{li2010multiple} %An theoretical analysis of upper bound of bit error rate (BER) and lower bound of on asymptotic multiuser efficiency for the LAS detector was also presented\cite{sunfamily}.
and Reactive Tabu search (RTS) algorithms and variants
%which have superior performances compared to LAS detectors because of the efficient local minima exit strategy 
(e.g. Layered Tabu search (LTS)\cite{srinidhi2011layered}, Random Restart Reactive Tabu search (R3TS)\cite{datta2010random}). Additionally, there are other algorithms proposed for large MIMO systems including Message passing technique based algorithms (e.g. Belief propagation (BP) detectors based on graphic model and Gaussian Approximation (GA)\cite{som2011low}\cite{som2010improved}\cite{narasimhan2014channel}\cite{goldberger2011mimo}), Probabilistic Data Association (PDA) based algorithms \cite{mohammed2009low}, Monte Carlo sampling technique based algorithms (e.g. Multiple Restart- Mixed Gibbs Sampling (MR-MGS) algorithm\cite{datta2013novel}) and Element based Lattice Reduction (LR) aided algorithms\cite{zhou2013element}.

 Considering MIMO detection from a Combinatorial Optimization (CO) problem viewpoint, as powerful tool for solving CO problems, methoheuristic algorithms\cite{blum2003metaheuristics} are good choices for designing large MIMO detectors, driven by demand of achieving acceptable performance with significantly lower computational complexity. Besides the metaheuristic algorithms that based on local search strategies which use trajectory methods based on single solution, another class of metaheuristic algorithms is defined as population based. The population based metaheuristic algorithms deal with a population of candidate solutions. Intrinsically the population based metaheuristic algorithms can provide wider and more efficient exploration of search space. The major population based metaheuristic algorithms includes Evolutionary Computation (EC) and Ant Colony Optimization (ACO). 
 
 Genetic algorithm (GA) is one kind of EC algorithms, which are designed for CO problems. GA mimics the natural evolution process of a population and is powerful tool in searching a solution that is close enough to global optimum\cite{Goldberg:1989:GAS:534133}.
 
 \section{Support Vector Regression}
 
 Firmly grounded in framework of statistical learning theory or VC (Vapnik--Chervonenkis) theory, the Support Vector (SV) technique has become a powerful tool to solve real world supervised learning problems such as classification, regression and prediction. The SV method is a nonlinear generalization of Generalized Portrait algorithm developed by Vapnik in 1960s\cite{vapnik1963pattern}\cite{vapnik1964note}, which can provide good generalization performance\cite{scholkopf2002learning}.

Interest in SV algorithms boosted since 1990s, promoted by the works of Vapnik and co-workers at AT\& T Bell laboratory\cite{boser1992training}\cite{guyon1993automatic}\cite{vapnik2013nature}\cite{cortes1995support}\cite{scholkopf1996incorporating}\cite{vapnik1996support}.
Moreover, the kernel based methods\cite{scholkopf2002learning} were proposed in order to extending the SV algorithms to nonlinear learning cases. The input data samples are mapped into some high dimensional feature space (also called Reproducing Kernel Hilbert Space (RKHS)) and then linear tools are applied to the feature mappings of the input data samples. This is equivalent to transforming the nonlinear learning tasks in the original space into the linear learning tasks in high dimensional feature space. The mathematical notion underlying kernel based methods is that of RKHS\cite{scholkopf2002learning}, in which the inner products of the feature mappings can be simply replaced by the computationally economical kernel functions. Because SV algorithms only deal with the inner products of the the feature mappings. Therefore, by kernel based methods, it is sufficient to use the specific kernel functions based on the RKHS discarding the actual structure of the feature space.   
 
Based on the principle of structural risk minimization\cite{vapnik1982estimation}, the $\epsilon$-SVR\cite{vapnik2013nature}\cite{smola2004tutorial} solves an original constraint optimization problem (primal objective problem) by transforming it into a Lagrange dual form (dual objective problem), which is a Quadratic Programming (QP) problem. 

Decomposition method is an efficient method to solve QP problems with large scale data sets, which decomposes a large QP problem into sub QP problems and solve them in an iterative manner\cite{osuna1997training}\cite{osuna1997improved}. Decomposition process is performed by sub set selection solver, which refers to a set of algorithms that separate the optimization variables into two sets S and N, S is the work set and N is the freezing set. In each iteration, only the optimization variables in the work set is updated while keeping other optimization variables fixed. Based on this principle, Sequential Minimal Optimization (SMO) algorithm\cite{platt1999fast} was proposed, which can solve the sub QP problems analytically instead of using the time consuming numerical steps. An important issue of the sub set selection solvers is the selection of the work set. One strategy is to choose Karush-Kuhn-Tucker (KKT) condition violators, ensuring the final converge\cite{osuna1997improved}. The SMO algorithm restricts the size of the work set to two. A method to train SVM without offset was proposed In\cite{steinwart2011training}, with the comparable performance to the SVM with offset. A set of sequential single variable work set selection strategies, which require $O(n)$ searching time are proposed. The optimal double variable work set selection strategy, which performs exhaustively searching, however, requires $O(n^{2})$ searching time. The authors demonstrate that with the combination of two different proposed single variable work set selection strategies, convergence can be achieved by a iteration time that is as few as optimal double variable work set selection strategy.

The mathematical foundation of kernel based methods is RKHS which is defined in complex domain, however most of the practitioners are dealing with real data sets. In communication and signal processing area, the channel gains, signals, waveforms etc. are all represented in complex form. Recently, a pure complex SVR \& SVM based on complex kernel was proposed in\cite{bouboulis2013complex}, which can deal with the complex data set purely in complex domain. The results in\cite{bouboulis2013complex} demonstrate the better performance as well as reduced complexity comparing to simply split learning task into two real case by real kernels.  
Based on this work, we derive a complexity-performance controllable detector for large MIMO systems based on a dual channel complex SVR (CSVR). The detector can work in two parallel real SVR channels which can be solved independently. Moreover, only the real part of kernel matrix is needed in both channels. This means a large amount of computation cost saving can be achieved.
Based on the discrete time MIMO channel model, in our regression model, this CSVR-detector
is constructed without offset, Therefore, for each real SVR without offset, 
%in principle, only one variable is needed to be updated in each iteration, In our scheme, a sequential single variable selection strategy is proposed. By this strategy, two variables can be updated at each iteration, with much smaller searching time.
Two types of combined single optimization variable selection strategy are proposed based on the work in \cite{steinwart2011training}. The proposed combined single optimization variable selection strategy can approximate optimal double optimization variable selection strategy. The former one can achieve as few as iteration time while enjoy significant speed gain in each iteration.
\section{Thesis Contribution}
\section{Thesis Outline}
%==========
\typeout{}
\resetdatestamp


\chapter{Theoretical Analysis of Channel Hardening Phenomenon}
\section{System Model}
%==========
\typeout{}
\resetdatestamp



\chapter{Complex Support Vector Preliminary Detector (CSVPD) for Large-Scale MIMO Uplink Systems}
Decomposition methods were proposed to solve this QP problem by decomposing it into sub QP problems and solving them iteratively\cite{platt1999fast}. Therefore, the computational intensive numerical methods can be avoided. Decomposition is performed by sub set selection solver, which refers to a set of algorithms that separate the optimization variables (Lagrange multipliers) into two sets S and N, S is the work set and N contains the remaining optimization variables. In each iteration, only the optimization variables in the work set is updated while keeping other optimization variables fixed. The Sequential Minimal Optimization (SMO) algorithm\cite{platt1999fast} is an extreme case of the decomposition solvers. An important issue of the sub set selection solvers is the selection of the work set. One strategy is to choose Karush-Kuhn-Tucker (KKT) condition violators, ensuring the final converge\cite{osuna1997improved}. The SMO algorithm restricts the size of the work set to two. A method to train SVM without offset was proposed In\cite{steinwart2011training}, with the comparable performance to the SVM with offset. A set of sequential single variable work set selection strategies, which require $O(n)$ searching time are proposed. The optimal double variable work set selection strategy, which performs exhaustively searching, however, requires $O(n^{2})$ searching time. The authors demonstrate that with the combination of two different proposed single variable work set selection strategies, convergence can be achieved by a iteration time that is as few as optimal double variable work set selection strategy.

The mathematical foundation of kernel based methods is RKHS which is defined in complex domain, however most of the practitioners are dealing with real data sets. In communication and signal processing area, the channel gains, signals, waveforms etc. are all represented in complex form. Recently, a pure complex SVR \& SVM based on complex kernel was proposed in\cite{bouboulis2013complex}, which can deal with the complex data set purely in complex domain. The results in\cite{bouboulis2013complex} demonstrate the better performance as well as reduced complexity comparing to simply split learning task into two real case by real kernels.  
Based on this work, we derive a complexity-performance controllable detector for large MIMO systems based on a dual channel complex SVR (CSVR). The detector can work in two parallel real SVR channels which can be solved independently. Moreover, only the real part of kernel matrix is needed in both channels. This means a large amount of computation cost saving can be achieved.
Based on the discrete time MIMO channel model, in our regression model, this CSVR-detector
is constructed without offset, Therefore, for each real SVR without offset, 
%in principle, only one variable is needed to be updated in each iteration, In our scheme, a sequential single variable selection strategy is proposed. By this strategy, two variables can be updated at each iteration, with much smaller searching time.
Two types of combined single optimization variable selection strategy are proposed based on the work in \cite{steinwart2011training}. The proposed combined single optimization variable selection strategy can approximate optimal double optimization variable selection strategy. The former one can achieve as few as iteration time while enjoy significant speed gain in each iteration.
\section{The Algorithm}
%==========
\typeout{}
\resetdatestamp

\newcommand\Dfrac[2]{\frac{\displaystyle #1}{\displaystyle #2}}
\newcommand{\mathBF}[1]{\mbox{\boldmath $#1$}}
\newcommand{\C}[1]{\mathBF{#1}}

\chapter{Mathematical Layout Styles}

\TeX{} does a marvelous job of setting mathematical formulas, most often
 choosing pleasing spacing.
However, on occasion one should intercede to improve the layout.
This chapter defines a few such occasions.
In addition, this chapter documents some features of the {\tt amsmath}
 package which overcome difficulties in typesetting some mathematical
 forms.
The {\tt amsmath} package is documented 
 in {\it The \LaTeX{} Companion} \cite{Goossens:1997}.

The modified setup is typeset as
\begin{equation}
  G(z) = \begin{cases}
           \Dfrac {P(z)}{1+z^{-1}} & \text{for $p$ even}, \\[1ex]
           P(z)                    & \text{for $p$ odd}.
         \end{cases}
\end{equation}

With the modified definitions, we get the following.
\def\hC#1{\C{\hat{#1}}\vphantom{\C{#1}}}           % hat vector
\def\htC#1{\C{\hat{\tilde{#1}}}\vphantom{\C{#1}}}  % hat, tilde vector
\def\tC#1{\C{\tilde{#1}}\vphantom{\C{#1}}}         % tilde vector
\begin{equation}
\begin{split}
  \C{d}^{(i)} &= \hC{v}^{(i)} - \htC{v}^{(i)} \\
  \C{n}^{(i)} &= \C{u}^{(i)} - \tC{v}^{(i)}
\end{split}
\end{equation}


%==========
\typeout{}
\resetdatestamp

\chapter{Tables}

\section{Tables in \LaTeX{}}

Tables of many different sorts can be made with \LaTeX{}.
This chapter gives suggestions on producing tables, along with a number of
 examples.

To illustrate these rules, here is a table and the \LaTeX{} input which was
 used to generate it.
\begin{table}
  \centering
  \Tcaption {Filter specifications}
  \label {T:FSpec}
  \small
  \vskip 1ex
  \renewcommand\arraystretch{1.1}
  \def~{\phantom{0}}
  \def\ExSp#1{\noalign{\vskip #1}}
  \begin{tabular}{cccccc}
    \hline\hline \ExSp{0.4ex}
    Taps  & Transition & Stopband   & Passband & Stop-band & Ultimate \\
    ($N$) & Band       & Weighting  & Ripple   & Rejection & Stop Band \\
          &            & ($\alpha$) & dB       & dB        & dB \\
    \ExSp{0.4ex} \hline \ExSp{0.4ex}
    ~8 &   &   &  0.06~ & 31 & 31 \\
    12 & A & 1 &  0.025 & 48 & 50 \\
    16 &   &   &  0.008 & 60 & 75 \\[1.2ex]
    12 &   &   &  0.04~ & 33 & 36 \\
    16 & B & 1 &  0.02~ & 44 & 48 \\
    24 &   &   &  0.008 & 60 & 78 \\[1.2ex]
    16 &   & 1 &  0.07~ & 30 & 36 \\
    24 &   & 1 &  0.02~ & 44 & 49 \\
    32 & \raisebox{1.5ex}[0pt]{C}
           & 2 &  0.009 & 51 & 60 \\
    48 &   & 2 &  0.006 & 50 & 66 \\[1.2ex]
    24 &   & 1 &  0.1~~ & 30 & 38 \\
    48 & D & 2 &  0.006 & 50 & 66 \\
    64 &   & 5 &  0.002 & 65 & 80 \\[1.2ex]
    48 &   & 2 &  0.07~ & 32 & 46 \\
    64 & \raisebox{1.5ex}[0pt]{E}
           & 5 &  0.025 & 40 & 51 \\
    \ExSp{0.4ex} \hline
  \end{tabular}
  \vskip 1ex
  \begin{tabular}{cc}
    Transition  & Normalized \\
    Code Letter & Transition Band \\
    \ExSp{0.4ex} \hline \ExSp{0.4ex}
    A  & 0.14~~ \\
    B  & 0.10~~ \\
    C  & 0.0625 \\
    D  & 0.043~ \\
    E  & 0.023~ \\
    \ExSp{0.4ex} \hline
  \end{tabular}
  \qquad
  \begin{minipage}[c]{2.1 in}
    \sloppy
    The normalized transition band is the width of the transition band
     normalized to $2\pi$; that is, $(\omega_s - \pi/2) / (2\pi)$.
  \end{minipage}
\end{table}


%========== Appendices
\appendix

%==========
\typeout{}
\resetdatestamp

\chapter{\LaTeX{} Macros}
\label{A:LaTeXmacros}

The \LaTeX{} commands and macros used in formatting the title page for
 this document are shown in this appendix.

\section{Thesis Preamble}

The commands used to create the title page for a thesis are shown
 below.
The McGill University crest is brought in via a macro \verb|McGillCrest|
 which allows for setting the size and colour of an imported PostScript
 file which contains the actual crest.
The title page also includes a red separator line.


%========== Bibliography
\typeout{}
\begin{singlespace}
  \bibliographystyle{IEEEtran}
  \bibliography{citation}
\end{singlespace}

\end{document}
